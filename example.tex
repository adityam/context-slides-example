\environment slides

% To enable the use of sample images in ConTeXt distrubtion
\setupexternalfigures[location={local,global,default}]

\starttext

\setvariables
    [metadata]
    [
      title={Creating slides in ConTeXt},
      author={ConTeXt user},
      date={Jan 13, 2013},
      location={ConTeXt Interactive Tutorial},
    ]

\startslide[title={First Slide}]

\startitemize
    \starthead {First bullet}
      Avoid using too many bullets. It is better to use paragraphs.
      \startitemize
        \item Especially nested bullet lists
        \item That make it hard to understand the main point
      \stopitemize
    \stophead

    \starthead {Second bullet}
      Nonetheless sometimes you are stuck at using bullets. In such cases, make
      sure that the bullets look nice.
    \stophead
\stopitemize

\stopslide

\startslide[title={Sample Slide}]

  \input knuth

\stopslide

\startslide[title={Some math}]

  Here is an example with display math with a long line preceeding it
  \startformula
    f(x) =  a_i x^i
  \stopformula
  and a long line following the display math. 

\stopslide

\startslide[title={Horizontal Image}]
  \startplacefigure
    \externalfigure[hacker][horizontal]
  \stopplacefigure
\stopslide

\startslide[title={Horizontal Image with text}]
  \input ward

  \startplacefigure
    \externalfigure[hacker][horizontal]
  \stopplacefigure
\stopslide

\startslide[title={Vertical Image}]
  \startplacefigure
    \externalfigure[mill][vertical]
  \stopplacefigure
\stopslide

\startslide[title={Vertical Image with text}]
  \startplacefigure[location=left]
    \externalfigure[mill][vertical]
  \stopplacefigure
  \input ward

\stopslide


\stoptext
